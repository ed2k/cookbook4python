\newcommand{\me}[1]{\mbox{\em{#1}}}
\newcommand{\mb}[1]{\mbox{\bf{#1}}}
\newcommand{\cmd}[2]{\mb{#1} #2}

\documentstyle[11pt]{article}
\addtolength{\textheight}{1.5in}
\addtolength{\topmargin}{-1in}
\addtolength{\textwidth}{1in}
\addtolength{\oddsidemargin}{-.5in}
\setlength{\parindent}{0mm}
\addtolength{\parskip}{1mm}

%\addtolength{\leftmargin}{-20mm}
%\addtolength{\rightmargin}{-20mm}
%\addtolength{\topmargin}{-20mm}

\input psfig

\documentstyle{twocolumn}
\begin{document}
\centerline{\psfig{figure=floaterlogo.ps}}
\begin{center}
\vspace{-6pt}
{\large{\bf{Reference for Floater Commands}}}\\
Version 1.2b2, \today\\
{\tiny Website: {\verb|http://www.floater.org/|}}
\end{center}
\vspace{-12pt}
\mbox{}\\ \cmd{accept}{
Accept declarer's claim.  Must be used by each defender for a claim to
be accepted.
}\\ \cmd{alert}{
``Alert some ... text ..." is equivalent equivalent to ``explain Alert!
some ... text ..." and is traditionally used to self-alert bids or
carding that is slightly unusual.  ``Redalert" should be used for highly
unusual bids or carding.
}\\ \cmd{autodeal}{
After a certain amount of time elapses at the end of a hand, Floater
automatically deals a new hand if four people are seated.  To control
the timing, ``autodeal `N'" sets the number of seconds to wait between
hands.  With no arguments, ``autodeal" toggles on and off whether new
hands are ever dealt automatically.  (By the way, you must be host of
the table for your autodeal setting to have any effect.)
}\\ \cmd{beep}{
You may beep someone at your table by doing ``beep `name'" (e.g., ``beep
Joe Smith"), or you may beep everyone at your table by just doing
``beep."  The bell will ring on the recipient's machine and everyone at
the table will see a message indicating who has beeped whom.
}\\ \cmd{beepAtMyTurn}{
There is an option that causes Floater to beep when it becomes your
turn to bid or play.  To toggle the described behavior, invoke this
command.  To turn it off, invoke this command with the argument ``no."
To turn it on, invoke this command with the argument ``yes."  See also
the options menu on the GUI.
}\\ \cmd{bid}{
Make a call.  For example, to bid 3 Notrump, do ``bid 3n".  Each strain
is represented by a single letter: c for clubs, d for diamonds, h for
hearts, s for spades, and n for notrump.  ``Bid p," ``bid pass," ``bid
x," ``bid xx," ``bid double," and ``bid redouble" have the obvious
meanings.  Unless you have customized the bindings (see ``bind"), the
word ``bid" may be omitted (``3n" is the same as ``bid 3n" and so on).
}\\ \cmd{bidButtons}{
If you are using the GUI, by default (unless your screen is small)
Floater displays, during the auction, a set of buttons for bidding.
To toggle the described behavior, invoke this command.  To turn it
off, invoke this command with the argument ``no."  To turn it on,
invoke this command with the argument ``yes."  See also the options
menu on the GUI.
}\\ \cmd{bind}{
Bind is used to create shorthand ways of specifying any command or
sequence of commands.  For example, ``bind typ say Thanks, pard!" would
allow you to subsequently use ``typ" as if it were a normal command.
Another example: Blue Club players might do ``bind 1c bid 1c; explain
Blue club: 17+ points, any distribution."
}\\ \cmd{bottom}{
If you are using the textual user interface, bottom is used to go to
the bottom of the scrolling talk window.  You may also use the scroll
command or the up and down array keys to select the portion of the
scrolling talk window you view.
}\\ \cmd{bug}{
Do ``bug `arbitrarily long one-line description'" and an email message
containing your description and a dump of some internal variables will
be sent to floater-bugs@priam.cs.berkeley.edu.
}\\ \cmd{bugs}{
Display the `BUGS' file.
}\\ \cmd{c}{
Play your lowest club.
}\\ \cmd{cards}{
Remind me what my cards were at the beginning of the hand.  May also
be used by kibitzers, with analogous meaning.
}\\ \cmd{cc}{
Invokes the textual user interface's convention card editor.  By
default you will be editing your side's convention card, or NS's if
you are not seated.  (``Editing" of a conventioning card other than
your own is really limited to viewing; changes are not allowed.)  The
three forms of the command are ``cc," ``cc NS," and ``cc EW."  If you are
using the GUI, ``cc" is equivalent to ``ccdump" because there is no
built-in editor for the GUI.
}\\ \cmd{ccdump}{
Show the specified convention card (e.g. ``ccdump NS").  If none is specified,
show my convention card if I am seated or NS's otherwise.  Furthermore,
you may also specify a range of line numbers to display (e.g. ``1-5" or
``23-" or ``-9" or just ``17").  Put the range first, if any, first;
so, ``ccdump -3" or ``ccdump 12-16 NS."
}\\ \cmd{ccload}{
Load a convention card from the named text file (e.g. ``ccload sam.cc")
and make it the convention card for my side.  You must be seated to
invoke the ccload command.
}\\ \cmd{ccsave}{
Save your current convention card to the named text file.
}\\ \cmd{changelog}{
Display the log of changes to Floater.
}\\ \cmd{children}{
Show who gets messages from me in the communication tree.  Primarily
intended for debugging.
}\\ \cmd{claim}{
As declarer, suggest to the defense that you are certain of some
number of tricks, and further play is not necessary.  The defenders
will be shown all four hands and may then use ``accept" or ``reject" and/or
continue playing.  Simply doing ``claim" claims the rest of the tricks;
``claim 3" would mean a claim to make 3 of the remaining tricks;
``claim -2" would be a claim to make all but 2 of the
remaining tricks; ``claim 0" means concede the rest; and ``claim +3"
would be a claim for the contract plus 3 overtricks.
}\\ \cmd{close}{
Stop hosting (or close) your table.  You must be hosting a table for
this to be allowed.
}\\ \cmd{competitive}{
Turns on competitive mode.  This is the default.  Results will be
reported by email for IMPs and MPs.  (Competitive mode is not relevant
to hearts or rubber bridge.)
}\\ \cmd{confusing}{
Display the `CONFUSING' file.
}\\ \cmd{copyright}{
Display copyright and warranty information.
}\\ \cmd{d}{
Play your lowest diamond.
}\\ \cmd{deal}{
Go on to the next hand.  May only be used by the host of the table.
}\\ \cmd{deiconifyIfBeeped}{
Normally when you are beeped by another player (see ``beep"), your
Floater window is deiconified if it had been iconified.  This applies
only to the GUI.  To toggle the described behavior, invoke this
command.  To turn it off, invoke this command with the argument ``no."
To turn it on, invoke this command with the argument ``yes."  See also
the options menu on the GUI.
}\\ \cmd{disconnect}{
Close all network connections.  You will become a ``lurker."  If you
are hosting a table at the time you issue the command, it is
equivalent to ``close."
}\\ \cmd{down}{
A variant of the ``claim" command.  For example, ``Down 2" is a claim
for 2 fewer tricks than you have contracted for.
}\\ \cmd{e (or east)}{
If the East seat is available, take it.
}\\ \cmd{east (or e)}{
If the East seat is available, take it.
}\\ \cmd{emailchange}{
This command may be used to inform the Floater login server of a change in
your email address.  (Such changes do not apply to the Floater mailing list.)
You must be logged in to use the command, and you should specify your
full email address as an argument to the command (e.g., ``emailchange
bob@aol.com").
}\\ \cmd{execute}{
Read a file and interpret the lines of the file as if they were typed
in on the command line at that moment.  E.g., ``execute ~/floater/foo."
}\\ \cmd{explain}{
Say something that everyone at the table except partner can hear.
This is typically used to explain conventional calls to the opponents
and kibitzers, among other things.
}\\ \cmd{find}{
Find information about people.  For example, ``find Joe Bob, elmo" will
attempt to find information about the players Joe Bob and elmo.  Any
number of names, separated by commas, is allowed.  Recent information
about a player's location, if available, is among the things displayed.
}\\ \cmd{follow}{
Play the card of the named rank.  E.g. ``follow k."  Unless you have
customized the bindings (see ``bind"), the word ``follow" may be omitted
(``7" is the same as ``follow 7" and so on).
}\\ \cmd{font}{
If you are using the GUI, you may change the font size for the matrix
and auction by doing ``font large," ``font medium," or ``font small." 
(To change the font for the talk window, use the ``talkfont" or
``talkfontsize" command.) 
}\\ \cmd{h}{
Play your lowest heart.
}\\ \cmd{hideAuction}{
If you are using the GUI, you may use this command to control when
during the first trick the auction is removed from view.  The default,
which can also be achieved by do `hideAuction -1' is to removed the
auction after trick one.  The command `hideAuction N' (for positive N)
will cause the auction to be hidden at the end of trick one or N
seconds after the end of the auction, whichever comes first.
}\\ \cmd{hideCommandLine}{
If you are using the GUI, you may have either a talk line by itself,
or a talk line and a command line.  Use this command with no arguments
to toggle between the two, or use it with an argument of ``yes" or ``no"
with the obvious meaning.  Note that commands may be executed on the talk
line by starting the line with a slash (e.g., ``/pass").
}\\ \cmd{hideMatrix}{
If you are using the GUI and have a small screen, Floater will, by
default, hide the matrix during the auction to give more room to other
things on the screen.  (Your cards appear at the top of the Floater
window.)  To toggle the described behavior, invoke this command.  To
turn it off, invoke this command with the argument ``no."  To turn it
on, invoke this command with the argument ``yes."  See also the options
menu on the GUI.
}\\ \cmd{host}{
Host a table.
}\\ \cmd{ip}{
Display the IP address and port number that others may use to connect
to you.
}\\ \cmd{join}{
Join a table.  You must specify the name of the host of the table you
wish to join, e.g., ``join moe".  (Alternatively, you may specify an IP
address and port separated by a colon, e.g., ``join 128.110.38.14:8765."
Or, to join a table hosted on your local machine, you may omit the IP
address and simply put the port number after a colon.)
}\\ \cmd{kibbitz}{
If you are sitting North, South, East, or West, this is the way to get up.
}\\ \cmd{last}{
Show the previous trick.
}\\ \cmd{lho}{
Say something to your left-hand opponent.  E.g. ``lho Is that for penalty?"
}\\ \cmd{login}{
Connect to the login server and transmit your account name and
password.  In return, the login server will check your password and
authenticate you to let you play under that account name.  In
addition, a list of tables currently in play will be sent to you,
allowing the ``tables" command to give up-to-date information.
}\\ \cmd{make}{
A variant of the ``claim" command.  With no arguments it is a claim to
make the contract exactly; with a positive integer argument it is a
claim to make a grand total of that number of tricks plus six.
}\\ \cmd{n (or north)}{
If the North seat is available, take it.
}\\ \cmd{newuser}{
Log in as a new user (see also ``login").
}\\ \cmd{noncompetitive}{
Turns off competitive mode.  You will play the same hands as you would
otherwise, but results will not be reported by email.  (Not relevant
for hearts or rubber bridge.)
}\\ \cmd{north (or n)}{
If the North seat is available, take it.
}\\ \cmd{note}{
Set the text that is listed alongside my table's name when people use
the ``tables" command.  For example, ``note Need 1!"  You must be host
of the table.
}\\ \cmd{opp}{
Say something just to your opponents.  Frequently one would use
``explain" instead so that the kibitzers could also hear.
}\\ \cmd{parent}{
Show my parent in the communication tree.  Primarily intended for debugging.
}\\ \cmd{password}{
Change the password for an account.
}\\ \cmd{previous}{
Show the previous hand.
}\\ \cmd{play}{
Play a card.  Cards are specified suit first, rank second.  For
example, ``play c3" would play the 3 of clubs.  Unless you have
customized the bindings (see ``bind"), the word ``play" may be omitted
(``c2" is the same as ``play c2" and so on).  In Hearts, this command is
also used to add or remove a card from the set of cards you wish to
pass to your opponent before the play of the hand.
}\\ \cmd{quit}{
Terminate Floater.
}\\ \cmd{randomplay}{
Toggle a switch that, when set, causes automatic random plays or bids
whenever it is your turn.  Intended for debugging.
}\\ \cmd{readme}{
Display the `README' file.
}\\ \cmd{redalert}{
``Redalert some ... text ..." is equivalent equivalent to ``explain Red Alert!
some ... text ..." and is traditionally used to self-alert bids or
carding that is highly unusual.  ``Alert" should be used for only slightly
unusual bids or carding.
}\\ \cmd{reject}{
Reject declarer's claim.  A rejection from either defender nixes the
claim regardless of what the other defender may have done.  Others at
the table are shown only that the claim was rejected, not by whom.
}\\ \cmd{retract}{
Declarer may use this command to retract a claim.
}\\ \cmd{review}{
Review the auction.
}\\ \cmd{rho}{
Say something to your right-hand opponent.  E.g. ``rho What is 2D?"
}\\ \cmd{s}{
If the south seat is available, try to sit down.  Otherwise, if you
are seated and it is your turn to play a card, play your lowest spade. 
}\\ \cmd{say}{
Say something to everyone present.  E.g. ``say Hello There!"
}\\ \cmd{score}{
Select the form of scoring for subsequent deals.  The valid uses of
the command are ``score IMP" and ``score MP."  See also ``competitive"
and ``noncompetitive."
}\\ \cmd{scroll}{
If you are using the textual user interface, scroll is used to change
which portion of the scrolling text window you view.  For example,
``scroll -10" takes you back 10 lines; ``scroll 20" takes you forward 20
lines.  You may also use the bottom command or the up and down array
keys to select the portion of the scrolling talk window you view.
}\\ \cmd{separateTalk}{
This command applies only to the GUI.  By default, the messages (e.g.,
from people talking) that Floater displays appear in a scrollable box
towards the bottom of the main Floater window.  If invoked with no
arguments, this command toggles whether that is true or whether
Floater has a separate window just for those messages.  It also may be
invoked with an argument of ``yes" or ``no," with the obvious meaning.
See also the options menu in the GUI.
}\\ \cmd{south (or s)}{
If the South seat is available, take it.
}\\ \cmd{spec}{
Become a double-dummy spectator.
}\\ \cmd{tables}{
List all tables.
}\\ \cmd{talkfont}{
If you are using the GUI, you may change the font family for the talk
window by doing, for example, ``talkfont times."  On some systems, not
all fonts are available in all sizes.
}\\ \cmd{talkfontsize}{
If you are using the GUI, you may change the font size for the talk
window by doing ``talkfontsize N" for any number N.  The number N is
interpreted as a size in points.  One point is roughly 1/72 inch.
}\\ \cmd{w (or west)}{
If the West seat is available, take it.
}\\ \cmd{warranty}{
Display copyright and warranty information.
}\\ \cmd{west (or w)}{
If the West seat is available, take it.
}\\ \cmd{who}{
List who is at the table I am at.
}\\ \cmd{whois}{
An IRC user suggested that ``whois" should be interchangeable with
``find" in Floater.  See find.
}\\ \cmd{.}{
Play your smallest card of the suit led.
}
\end{document}